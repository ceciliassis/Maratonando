\documentclass{article}
\usepackage[hmargin=1in,vmargin=1in]{geometry}
\usepackage{listings}
\usepackage{color}

% For better handling of unicode (Latin characters, anyway)
\usepackage{lmodern}        % Usa a fonte Latin Modern
\usepackage[T1]{fontenc}    % Selecao de codigos de fonte.
\usepackage[utf8]{inputenc} % Codificacao do documento (conversão automática dos acentos)

\lstset{
    numbers=left,                   % where to put the line-numbers
    numberstyle=\small \ttfamily \color[rgb]{0.4,0.4,0.4},
                % style used for the linenumbers
    showspaces=false,               % show spaces adding special underscores
    showstringspaces=false,         % underline spaces within strings
    showtabs=false,                 % show tabs within strings adding particular underscores
    frame=lines,                    % add a frame around the code
    tabsize=4,                        % default tabsize: 4 spaces
    breaklines=true,                % automatic line breaking
    breakatwhitespace=false,        % automatic breaks should only happen at whitespace
    basicstyle=\ttfamily,
    %identifierstyle=\color[rgb]{0.3,0.133,0.133},   % colors in variables and function names, if desired.
    keywordstyle=\color[rgb]{0.133,0.133,0.6},
    commentstyle=\color[rgb]{0.133,0.545,0.133},
    stringstyle=\color[rgb]{0.627,0.126,0.941},
}

\begin{document}

\section*{codes/contagem.cpp}
\lstinputlisting[language=C++]{"codes/contagem.cpp"}

\section*{codes/dfs.cpp}
\lstinputlisting[language=C++]{"codes/dfs.cpp"}

\section*{codes/falha.cpp}
\lstinputlisting[language=C++]{"codes/falha.cpp"}

\section*{codes/passelivre.cpp}
\lstinputlisting[language=C++]{"codes/passelivre.cpp"}

\section*{codes/promocao.cpp}
\lstinputlisting[language=C++]{"codes/promocao.cpp"}

\section*{codes/resolver.cpp}
\lstinputlisting[language=C++]{"codes/resolver.cpp"}

\section*{codes/streams.cpp}
\lstinputlisting[language=C++]{"codes/streams.cpp"}

\end{document}
